\documentclass{article}
\usepackage[a4paper, total={6in, 10in}]{geometry}

\usepackage{multicol}
\usepackage{enumitem}

\usepackage{xcolor}
\usepackage{listings}
\usepackage{xparse}

\NewDocumentCommand{\codeword}{v}{
    \texttt{\textcolor{blue}{#1}}
}

\NewDocumentCommand{\cmd}{v}{
    \textit{\textcolor{purple}{#1}}
}

\title{World of Deez}
\author{Joe}
\date{19th November 2021}

\begin{document}
\maketitle

    \begin{multicols}{2}
        \section{User Level Description}
        In this game you play Among Us. deez nuts
        
        \section{Implementation}

        talk about implementation
        
            % \subsection{Game}
            
            % \subsection{World}
            
            %     \subsubsection{Rooms}
            
            % \subsection{Entities}
            
            %     \subsubsection{Actions}
                
            %     \subsubsection{Behaviours}
            
            % \subsection{Commands}
            
            % \subsection{Campaign}

        \section{Base Tasks}

            This section describes how I implemented the basic task requirements.

            \begin{itemize}[leftmargin=*]
                \item \textit{``The game has several locations the player can walk through.''}
                
                    I ended up adding $10$ locations into my game:
                    \begin{itemize}[leftmargin=*]
                        \item \textbf{City Centre}: Centre of the city connecting major areas.
                        \item \textbf{Apartments}: This is the player's residence.
                        \item \textbf{Street}: This is the main city street connecting important buildings.
                        \item \textbf{Shop}: The local city shop where the player frequents to get necessary items.
                    \end{itemize}

                \item \textit{``There are items in some rooms that may or may not be picked up by players.''}
                
                    To achieve this, I considered all entities to be items which may or may not be picked up by other entities, each entity has its own \codeword{Inventory} which is in effect a list of other entities which it is holding.
                
                \item \textit{``Each item has a weight and the player can only carry items up to a certain weight.''}
                
                    To do this, I added a new private field \codeword{weight} of type double to the \codeword{Entity} class, which is used to store the entity's weight. It is a double as I wanted access to fractional weights (say $0.01$kg) and I wanted to have access to \codeword{Stream::mapToDouble} for summations.

                    For the second part, I made it so each \codeword{Inventory} has a maximum weight it can store, which by default is set to $0$ as each entity has an inventory but may not necessarily have the ability to store anything.

                    When putting anything in an inventory, we check that the following is satisfied:
                    $$
                        \textsf{currentWeight} + \textsf{itemWeight} \le \textsf{maxWeight}
                    $$
                
                \item \textit{``Player can win.''}
                
                    The player may win by completing the main story mission which sets a flag that the game has been completed, the player may choose to keep playing in the open world or run \cmd{win} to end the game.

                \item \textit{``There is a command} \cmd{back} {which takes you back to the last room.''}
                
                    I added two new private fields to the player entity, which were \codeword{previousRooms} and \codeword{retreatingDirection}, these are used to store the path back through the room and the direction which we need to go to get back there respectively. The direction is stored in order to run a check whether the player can actually go back in the direction they intend to, to verify this, the `retreating direction' is used to call the method \codeword{canLeave} on the current room the player is in.

                \item \textit{``Add at least four new commands.''}
                
                    %
            \end{itemize}

        \section{Challenge Tasks}

            This section describes how I implemented the challenge tasks and what I did in addition.

            \begin{itemize}[leftmargin=*]
                \item \textit{``Add characters to your game.''}
                
                    %

                \item \textit{``Extend the parser.''}
                
                    I entirely replaced how the parser worked, I began by implementing a basic model of \codeword{Command}, it took a few iterations but I settled on providing Regex \codeword{Pattern}s.

                    This approach has several benefits:

                    \begin{itemize}[leftmargin=*]
                        \item Powerful Regex at low performance cost due to the low number of commands.
                        \item Ability to have named capture groups which are then interpreted as arguments.
                    \end{itemize}

                    For each known \codeword{Command}, I would create a \codeword{Matcher} for each \codeword{Pattern}, execute it against the arbitrary command from the user and then pass it into a wrapper class \codeword{Arguments} which lets me safely pull out named groups, directions or any other argument type I need.
                
                \item \textit{``Add a magic transporter room.''}
                
                    To do this, I added a \codeword{RoomWormHole} which I made implement \codeword{EventEntityEnteredRoom} to listen for when any entities entered the room, as soon as one is detected, a short animation is played and a room is selected at random to teleport the user to. Allowing the user to go to \textbf{any room} may interfere with my story so I chose to only spawn the user at any outside areas of the map.
                
                \item \textit{``Implement a terminal emulator.''}
            
                    I implemented this by creating a new class \codeword{TerminalEmulator} which implements the \codeword{IOSystem} so that it could be easily slotted into the game. 
                    %
                
                \item \textit{``Give NPCs interactive dynamic dialogue.''}
                
                    %

                \item \textit{``Add a system for managing world events.''}
                
                    %
            \end{itemize}

        \section{Code Quality}
        %

        \section{Walkthrough}
        %

        \section{Known Issues}
        
            \begin{itemize}
                \item testing
            \end{itemize}

    \end{multicols}

    \begin{thebibliography}{9}
        \bibitem{something}
        Some sort of references
    \end{thebibliography}        

    \begin{multicols}{2}
        \section{First Section}
        Lorem ipsum dolor sit amet, consectetur adipiscing elit. Proin congue leo nec ligula euismod, sit amet efficitur mi condimentum. Nulla pulvinar, justo sed sollicitudin euismod, mi velit vestibulum diam, et mollis odio felis id nulla. Nam id velit nec lacus fringilla cursus. Nulla quis tempus erat, a dignissim urna. Curabitur justo risus, varius vel iaculis finibus, eleifend quis metus. Proin id bibendum purus, non elementum ante. Vestibulum eu orci non erat luctus iaculis. Suspendisse potenti. Praesent libero leo, condimentum eu mauris et, mollis efficitur ligula. Duis eget nisi erat. Donec urna ipsum, convallis at faucibus non, maximus quis orci. Donec vitae neque ac dui tempor scelerisque id faucibus nulla. Nullam vel suscipit metus. Mauris vel semper nisl, eu elementum arcu.

        Vivamus in leo et enim laoreet tempor. Phasellus feugiat volutpat neque, a faucibus justo fringilla eleifend. Pellentesque condimentum ipsum nec justo blandit aliquam ac et nisl. Donec placerat et dui ac varius. Integer faucibus odio est, nec hendrerit ex scelerisque ac. In viverra tortor id aliquet eleifend. Sed ut dignissim ligula. Aenean dapibus ornare venenatis.

        \section{Second}
        Nulla vitae sodales tellus. Donec vel placerat velit. Sed ac consequat magna. Maecenas lacus magna, consequat eget dictum a, accumsan non nisl. Pellentesque vel mauris ullamcorper, pellentesque ipsum et, congue nisl. Aliquam faucibus nulla aliquam pulvinar ultricies. In ut odio id lectus tristique sodales eu ac ante. Donec commodo commodo scelerisque. Praesent ullamcorper, lacus non malesuada dignissim, tortor lectus elementum mauris, eget tempus ante nunc quis lectus. Sed gravida metus et urna tempus porta. Lorem ipsum dolor sit amet, consectetur adipiscing elit.

        Sed lorem metus, accumsan ut erat ut, vestibulum porta lorem. Vestibulum ante ipsum primis in faucibus orci luctus et ultrices posuere cubilia curae; Cras nec nisi mauris. Integer nisi nisi, gravida sed sodales quis, dignissim eu nisi. Vestibulum sollicitudin mattis ligula, vitae rutrum magna rhoncus ut. Lorem ipsum dolor sit amet, consectetur adipiscing elit. Etiam consectetur odio id diam fermentum, malesuada vehicula sapien accumsan. Fusce auctor quam sit amet imperdiet venenatis. Duis gravida pellentesque nibh, nec luctus mauris. Cras ipsum libero, rhoncus nec pretium non, hendrerit a sem. Fusce blandit dapibus est, vitae fermentum lectus mattis eu. Morbi et pretium nisi. Donec congue, nunc quis posuere hendrerit, elit risus vulputate velit, ut tempus nisi ipsum quis sapien. Class aptent taciti sociosqu ad litora torquent per conubia nostra, per inceptos himenaeos. Nunc libero lorem, feugiat sit amet erat ut, rutrum scelerisque purus.

        \begin{itemize}
            \item testing
                amogus!
            \item testing
        \end{itemize}

        Ut efficitur odio non tincidunt volutpat. Vivamus et fermentum libero, sit amet volutpat elit. Etiam egestas orci non pharetra facilisis. Mauris faucibus turpis vel purus luctus aliquam. Quisque vehicula felis id nulla mollis aliquet. In hac habitasse platea dictumst. Sed aliquet elementum nisl sed efficitur. In lobortis leo purus, nec pretium nunc suscipit sed. Aenean pretium porttitor est. Nam mollis eget quam et facilisis. Phasellus tincidunt tellus eu sapien gravida laoreet. Lorem ipsum dolor sit amet, consectetur adipiscing elit. In pellentesque convallis euismod. Sed ut lobortis enim. Aliquam a felis metus. Quisque id nibh tincidunt, tincidunt urna sed, interdum risus.

        Curabitur velit dui, luctus ut tincidunt vel, condimentum ut nisi. Ut porttitor odio nisl, non facilisis turpis molestie non. Sed consectetur ipsum eu rhoncus bibendum. In iaculis scelerisque semper. Nunc ac nisi nec justo tempus sollicitudin non sed diam. Integer lacinia sit amet orci vitae posuere. Aliquam vehicula sapien id turpis tempor malesuada rhoncus sed risus. Cras et scelerisque turpis, sed tristique nisi. Sed id dui sed lorem feugiat egestas et id nisl. Nunc ut venenatis mauris.

        Etiam vulputate, sem non laoreet facilisis, leo odio viverra lorem, at elementum quam ipsum et nunc. Pellentesque a augue volutpat, ultricies ligula vel, egestas felis. Vestibulum ante ipsum primis in faucibus orci luctus et ultrices posuere cubilia curae; Phasellus tincidunt quam justo, at condimentum purus rhoncus nec. Vivamus vel nisi sit amet ex mattis viverra quis ac leo.
    \end{multicols}
\end{document}
